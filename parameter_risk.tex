%%%%%%%%%%%%%%%%%%%%%%%%%%%%%%%%%%%%%%%%%%%%%%%%%%%%%%%%%%%%
%%  This Beamer template was created by Cameron Bracken.
%%  Anyone can freely use or modify it for any purpose
%%  without attribution.
%%
%%  Last Modified: January 9, 2009
%%

\documentclass[10pt, xcolor=x11names, compress, handout]{beamer}

%% General document %%%%%%%%%%%%%%%%%%%%%%%%%%%%%%%%%%
\usepackage{graphicx}
\usepackage[export]{adjustbox}
\usepackage{tikz}
\usepackage[T1]{fontenc}
\usepackage[latin1]{inputenc}
\usepackage{hyperref}
\usepackage{xcolor}
\usetikzlibrary{decorations.fractals}
%%%%%%%%%%%%%%%%%%%%%%%%%%%%%%%%%%%%%%%%%%%%%%%%%%%%%%


%% Beamer Layout %%%%%%%%%%%%%%%%%%%%%%%%%%%%%%%%%%
\usetheme{Berkeley}
\usecolortheme{sidebartab}

\definecolor{ltblue}{rgb}{0.61328125, 0.875, 0.92578125}
\definecolor{dkblue}{rgb}{0, 0.171875, 0.46484375}
\definecolor{mygray}{rgb}{0.74609375, 0.74609375, 0.74609375}
\definecolor{turq}{rgb}{0, 0.5390625, 0.69921875}


\setbeamertemplate{footline}{%
	\hfill\usebeamertemplate***{navigation symbols}
	\hspace{1cm}\insertframenumber{}/\inserttotalframenumber
}

\setbeamercolor{frametitle}{bg=dkblue}
\setbeamercolor{sidebar}{bg=dkblue}
\setbeamercolor{logo}{bg=dkblue}
\setbeamercolor*{item}{fg=turq}
\setbeamercolor{title}{bg=mygray, fg=black}


%%%%%%%%%%%%%%%%%%%%%%%%%%%%%%%%%%%%%%%%%%%%%%%%%%

\title{A Note on Parameter Risk}
\subtitle{Gary Venter \& \\
	Rajesh Sahasrabuddhe \textit{(in absentia})}
\date{
	CAS Annual Meeting 2016\\
	Session Code:P-5\\
	Wednesday, November 16\\
	8:00 AM - 9:15 AM	
	}

\begin{document}
\section{Main Part}
%\logo{\includegraphics[height=0.8cm]{chartis.eps}\vspace{20pt}}
\begin{frame}
\maketitle
\end{frame}
%\addtobeamertemplate{frametitle}{}{%
%\begin{textblock*}{100mm}(.85\textwidth,-1cm)
%\includegraphics[height=1cm,width=2cm]{chartis.eps}
%\end{textblock*}}

%-----------------------------------------------------------------------------------------------------------------------
\section{Agenda}
\begin{frame}{Agenda}
\begin{itemize}
	\item<1-> Why is parameter risk important?
	\item<2-> Where does parameter risk exist?
	\item<3-> How do I quantify parameter uncertainty?
	\item<4-> How can parameter risk be included in actuarial models?		
\end{itemize}
\end{frame}
%-----------------------------------------------------------------------------------------------------------------------

%-----------------------------------------------------------------------------------------------------------------------
\section{The Need to Include Parameter Risk}
\begin{frame}{Why is Parameter Risk Important?}
	\begin{enumerate}
		\item<1-> A form of systematic risk
		\begin{itemize}
			\item<2-> Does not diversify with volume (exacerbated by volume)
			\item<2-> But may diversify across products / portfolios
		\end{itemize}
		\item<3-> Inherent in all actuarial models		
	\end{enumerate}
\end{frame}
%-----------------------------------------------------------------------------------------------------------------------

%-----------------------------------------------------------------------------------------------------------------------
\section{Sources of Parameter Risk}
\begin{frame}{Where Does Parameter Risk Exist --  Basically Wherever You Estimate Parameters}
	\begin{enumerate}
		\item<1-> Regression Models (Least Squares)
			\begin{itemize}
				\item<1-> Cost $\thicksim$ Time (i.e. Trend)
				\item<1-> Cost $\thicksim$ Claim features (i.e. Predictive Modeling)
			\end{itemize}
		\item<2-> MLE -- Such as Loss distributions, Nonlinear Models
			\begin{itemize}
				\item<2-> Frequency models
				\item<2-> Severity models
				\item<2-> Aggregate loss models
			\end{itemize}
		\item<3-> Bayesian Estimation
		\begin{itemize}
			\item<2-> Traditionally done using conjugate priors
			\item<2-> Now simulate sample of parameter sets by MCMC $\ne$ 1100 1100
			\item<2-> Don't need to find conjugate priors -- any prior can work
		\end{itemize}		
	\end{enumerate}
\end{frame}
%-----------------------------------------------------------------------------------------------------------------------

%-----------------------------------------------------------------------------------------------------------------------
\section{Estimation of Parameter Risk}
\begin{frame}{How Do I Quantify Parameter Uncertainty?}
	\begin{enumerate}
		\item<1-> Regression Models
		\begin{itemize}
			\item<1-> Combines variance of the residuals and variances of each independent variable
			\item<1-> Each parameter estimate then t-distributed with combined variance
			\item<1-> Similarly the distribution of the fitted points is also t-distributed
		\end{itemize}
		
		\item<2-> MLE Estimation Uncertainty
		\begin{itemize}
			\item<2-> Large sample sizes: parameter estimates $\thicksim$ Normal\\
				Mean = MLE, minimizing negative loglikelihood (NLL)\\
				Normal covariance matrix via Information Matrix: matrix of all combinations of 2nd partial derivatives of the NLL	
			\item<2-> With limited sample size often better to use same mean and covariance matrix but assume each parameter is gamma distributed -- with normal copula for correlations
		\end{itemize}	
		
			\item<3-> Bayesian Estimation
					\begin{itemize}
			\item<3-> Posterior distributions for the parameters, including correlations, come out of the process directly, either in closed form or simulated
		\end{itemize}	


	\end{enumerate}
\end{frame}
%-----------------------------------------------------------------------------------------------------------------------

%-----------------------------------------------------------------------------------------------------------------------
\section{Consideration of Parameter Risk}
\begin{frame}{How Can Parameter Risk Be Included in Actuarial Modeling?}
	\begin{itemize}
		\item<1-> For most applications, model uncertainty is developed via simulation of the distribution of residuals
		\item<2-> BUT: The simulation often assumes that the parameters are known with certainty
		\item<3-> It is straightforward to FIRST simulate the parameters and THEN simulate the outcome of interest
		\item<3-> Even if the process risk diversifies away, the total risk does not, due to including parameter uncertainty
	\end{itemize}
\end{frame}
%-----------------------------------------------------------------------------------------------------------------------

%-----------------------------------------------------------------------------------------------------------------------
\begin{frame}{Questions}
	Venter, Gary G., and Rajesh V. Sahasrabuddhe, "A Note on Parameter Risk," Variance 9:1, 2015, pp. 54-63.
\end{frame}
%-----------------------------------------------------------------------------------------------------------------------





\end{document}
